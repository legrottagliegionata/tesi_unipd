\section{Conoscenze acquisite con lo \textit{stage}}
Prima di iniziare il progetto, mi sono posto degli obiettivi personali che ho riportato nella sezione \ref{obiettivi-personali}.\\
Ora che lo \textit{stage} è terminato, posso dire che tutti gli obiettivi sono stati raggiunti.\\
Ora sono in grado di utilizzare Gradle per la gestione dei progetti, partendo dalla creazione di un progetto vuoto,
riuscendo a configurarlo in modo da poterlo compilare, testare e distribuire.\\
Ho imparato a scrivere un \textit{plugin} per Gradle, riuscendo a comprendene il funzionamento e le potenzialità, riuscendolo 
ad utilizzare per altri progetti in corso.\\
Ho imparato a utilizzare i \textit{database} a grafo, e a comprendere quando è meglio utilizzarli rispetto ai \textit{database} relazionali.\\
Penso che sia uno strumento molto potente, e che in alcuni casi può essere molto utile, ma non credo che possa sostituire i \textit{database} relazionali.\\
Ho provato le ultime versioni di Angular e delle librerie che solitamente uso per lo sviluppo delle \textit{web-app}, 
e ho potuto valutare se è il caso di aggiornare i progetti che ho sviluppato in passato, e se è il caso di utilizzarle per i progetti futuri.\\
Ho scoperto che le ultime versioni di Angular sono molto più performanti rispetto alle precedenti, e che sono state introdotte delle funzionalità
molto utili, come ad esempio il nuovo \textit{control flow}, che ti permette di caricare parti di \textit{template} in base a determinate condizioni e in 
modo \textit{lazy}, riducendo il numero di elementi nel \textit{DOM} e migliorando le prestazioni.\\
Questa funzionalità ci ha aiutato a risolvere un problema di prestazioni in una \textit{web-app} che abbiamo sviluppato in passato.\\