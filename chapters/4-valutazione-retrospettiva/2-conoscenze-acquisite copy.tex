\section{Conoscenze acquisite con lo \textit{stage}}
Prima di iniziare il progetto, mi sono posto degli obiettivi personali, 
elencati nella sezione \ref{obiettivi-personali}. Ora, al termine dello \textit{stage}, 
posso affermare con soddisfazione che ho raggiunto tutti questi obiettivi.

Durante lo \textit{stage}, ho acquisito competenze nell'uso di Gradle per la gestione dei progetti. 
Ora sono in grado di creare un progetto da zero, configurarlo per la compilazione, il \textit{testing} e la distribuzione. \\
Ho anche imparato a scrivere un \textit{plugin} per Gradle, comprendendone il funzionamento e le potenzialità, 
e ho già iniziato ad applicarlo ad altri progetti in corso.

Un'altra competenza che ho acquisito riguarda l'uso dei \textit{database} a grafo. 
Ho compreso quando è vantaggioso impiegarli rispetto ai \textit{database} relazionali. 
Ritengo che siano uno strumento molto potente e utile in determinati contesti, 
anche se non sostituiscono completamente i \textit{database} relazionali.

Ho anche ho acquisito una buona comprensione dell'impiego dei \textit{database} a grafo, 
apprezzandone la potenza e l'utilità in specifici scenari. \\
Ho imparato a riconoscere i contesti in cui 
i \textit{database} a grafo eccellono, specialmente in situazioni dove le relazioni tra i dati sono complesse 
e interconnesse, come nelle reti sociali o nelle analisi di percorsi. Tuttavia, ho anche riconosciuto che 
i \textit{database} relazionali mantengono la loro superiorità in ambienti dove le strutture dati sono più 
tabellari e le relazioni meno intricate.

Infine, ho esplorato l'ultima versione di Angular e delle relative librerie per lo sviluppo di \textit{web-app}. 
Questo mi ha permesso di valutare l'aggiornamento dei progetti passati e l'adozione di queste tecnologie nei progetti futuri. 
Ho notato un significativo miglioramento delle prestazioni nelle versioni più recenti di Angular, grazie anche all'introduzione 
di funzionalità innovative come il nuovo \textit{control flow}. Questo meccanismo, che permette il caricamento condizionale 
e \textit{lazy} di parti di \textit{template}, riduce il numero di elementi nel \textit{DOM}, migliorando le prestazioni. \\
Ho applicato con successo questa funzionalità per risolvere un problema di prestazioni in una \textit{web-app} precedentemente sviluppata, 
ottenendo risultati notevoli.