\section{Lo \stage{} per \azienda}

Gli \stage{} rappresentano un pilastro fondamentale nella strategia di {\azienda}, svolgendo molteplici funzioni cruciali. 
Primo fra tutti, offrono l'opportunità di dedicarsi a progetti innovativi che, a causa di limitazioni di \textit{budget} o di tempo, 
non troverebbero spazio nelle attività lavorative ordinarie. I \textit{team} di sviluppo di {\azienda} sono primariamente impegnati 
nel mantenimento e nello sviluppo dei prodotti esistenti, lasciando poco margine per la ricerca e lo sviluppo di nuove idee. 
Questo compito solitamente è affidato al \textit{team} di ricerca e sviluppo, che, nonostante la sua competenza, è spesso limitato dalla scarsità 
di risorse umane per affrontare tutte le richieste innovative. In questo contesto, lo \stage{} diventa una soluzione efficace, 
consentendo lo sviluppo di prototipi e la conduzione di ricerche senza gravare eccessivamente sulle risorse aziendali.

L'investimento principale per l'azienda risiede nel tempo dedicato dal \textit{tutor} aziendale alla guida e al supporto dello stagista. 
Pertanto, una selezione accurata sia del \textit{tutor} sia dello stagista è essenziale per il successo del progetto e il raggiungimento 
degli obiettivi entro i tempi stabiliti.

\noindent Da anni, {\azienda} ha instaurato una proficua collaborazione con l'Università di Padova, accogliendo studenti per lo 
svolgimento di tesi di laurea e \stage. Questi periodi di formazione in azienda sono spesso orientati verso l'assunzione: negli 
ultimi anni, più di 140 stagisti sono entrati a far parte del team di {\azienda} a seguito del loro \stage.

Per lo studente, lo \stage{} si configura come un'esperienza formativa di grande valore. Consente di applicare concretamente le 
conoscenze teoriche acquisite durante il percorso di studi in un contesto lavorativo reale, favorendo al contempo l'acquisizione 
di nuove competenze e conoscenze. Questo periodo rappresenta un momento di verifica importante per lo studente, che ha l'opportunità 
di confrontarsi con il mondo del lavoro e di valutare se l'attività svolta corrisponde alle proprie aspettative e aspirazioni professionali future.

