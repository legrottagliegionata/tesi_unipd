\section{La proposta di \stage}

Un obiettivo strategico futuro per {\azienda} consiste nella modularizzazione dei propri prodotti \textit{software}. 
L'intento è quello di realizzare moduli che possano essere utilizzati in maniera autonoma e che, al contempo, 
siano facilmente integrabili con altre soluzioni. Questa direzione strategica implica un significativo aumento
del numero di progetti e delle interdipendenze tra essi, rendendo essenziale un'efficace gestione e monitoraggio di tali dipendenze.

In risposta a questa esigenza, il dipartimento di Ricerca e Sviluppo di {\azienda} ha intrapreso l'iniziativa 
di sviluppare un sistema avanzato per la raccolta, il monitoraggio e l'analisi delle dipendenze \textit{software}. 
L'obiettivo è quello di garantire un controllo accurato sulle versioni delle dipendenze collegate alle diverse release dei prodotti, 
contribuendo così a soddisfare i requisiti non funzionali in termini di sicurezza e affidabilità delle soluzioni offerte.

Nell'ambito di questa iniziativa, la proposta di \stage{} ha riguardato lo sviluppo di un \textit{\gls{prototipo}} per tale strumento. 
Il focus del progetto di \stage{} è stato lo sviluppo di un sistema in grado di raccogliere informazioni dettagliate sulle dipendenze 
direttamente dai sistemi di \textit{build} utilizzati per la compilazione dei prodotti, specificatamente \gls{Gradle} e \gls{Npm}. 
Questo prototipo rappresenta un passo fondamentale verso l'implementazione di una soluzione completa per la gestione delle dipendenze.

\subsubsection*{Il \textit{plugin} Gradle}
Il primo prodotto che l'azienda intendeva realizzare tramite lo \stage{} è un \textit{plugin} per Gradle che raccoglie informazioni sulle dipendenze dei progetti java, 
analizzando l'albero delle dipendenze generato da Gradle, e dei progetti npm analizzandone il file \textit{package-lock.json} generato
durante l'installazione dei pacchetti.
Il \textit{plugin} doveva poter essere pubblicato su un \textit{repository} interno aziendale, e doveva essere utilizzabile da tutti i progetti di {\azienda}.
Per essere più flessibile, doveva poter essere configurato specificando i nomi dei progetti npm da analizzare
ed i riferimenti al \textit{backend} per l'invio delle informazioni raccolte.\\

\subsubsection*{Il \textit{backend}}
Il secondo prodotto atteso era un \textit{backend} per il salvataggio e l'analisi delle informazioni raccolte. 
Il \textit{backend} doveva esporre dei servizi \textit{REST} per l'interrogazione del sistema, e doveva essere sviluppato utilizzando
il linguaggio di programmazione Java. Non è stato richiesto l'utilizzo di un \textit{framework} specifico, ma è stato lasciata
la libertà di scegliere il \textit{framework} più adatto al progetto.\\
Il \textit{plugin} dopo aver raccolto le informazioni
le invia al \textit{backend} che,  dopo averle elaborate, le salva in un \textit{database} a grafo.
Anche in questo caso è stato richieso un file di configurazione per specificare i riferimenti al \textit{database} a grafo, e le credenziali
per l'autenticazione ai servizi \textit{REST}.\\

\subsubsection*{Il \textit{frontend}}
Infine, un'interfaccia grafica realizzata tramite una \textit{web-app} per la visualizzazione delle informazioni raccolte.
Le specifiche per il \textit{frontend} erano molto generiche, ma per essere in linea con le tecnologie utilizzate in azienda,
è stato richiesto di utilizzare il \textit{framework Angular} per lo sviluppo.\\
