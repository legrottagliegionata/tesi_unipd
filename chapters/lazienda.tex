\chapter{L'azienda}
\label{cap:lazienda}


\section\azienda

\subsection{Descrizione}
L'azienda {\azienda}, fondata nel 1984, offre servizi di consulenza e sviluppo di software. 
Si è distinta nell'ideazione, costruzione e implementazione di strumenti software per oltre 2500 imprese, 
molte delle quali all'estero. \\
Una delle sue qualità distintive è l'attenzione verso i clienti, 
con vari uffici in regioni come Veneto, Lombardia, Emilia-Romagna, Friuli-Venezia Giulia, Toscana, Puglia e Campania, 
impiegando oltre 600 professionisti. \\
La sede centrale per la ricerca e sviluppo (CSV) si trova a Grisignano di Zocco (VI) e ospita più di 200 collaboratori. 
Qui, gruppi di sviluppatori 
e tecnici lavorano insieme per assicurare servizi affidabili e soluzioni software su misura. 
\newpage

\subsection{Organizzazione dell'azienza e i suoi prodotti}
{\azienda} è suddivisa in \textit{Business Unit} (BU), una parte di un'azienda che opera in modo autonomo o semi-autonomo, 
con la propria visione, \textit{mission}, obiettivi e strategie. Essa ha una propria \textit{leadership} e una struttura 
organizzativa separata, ed è responsabile del proprio profitto e perdite. \\
Le BU possono focalizzarsi su 
specifici mercati geografici, gruppi di clienti o linee di prodotti, permettendo all'azienda di essere 
più agile e rispondere meglio alle esigenze del mercato e dei clienti.\\
In {\azienda} BU sono 11 e, come rappresentate in figura \ref{fig:organizzazione-azienda}, si suddividono in:
\begin{itemize}
  \item \textbf{JGALILEO:} ha sviluppato l’\gls{ERP} –  Jgalileo, il sistema gestionale completo che consente alle imprese di monitorare e governare i flussi aziendali in modo semplice ed efficace, grazie a workflow condivisi e informazioni univoche e coerenti. Il software gestionale ERP Jgalileo si rivolge a tutte le aziende produttive e commerciali di ogni dimensione, dalla piccola azienda al grande gruppo aziendale internazionale, grazie anche alla gestione accurata delle fiscalità estere;
  \item \textbf{NEXTBI:} specializzata in \textit{Information Technology} e consulenza direzionale, con un focus nelle aree \textit{marketing}, vendite, \textit{retail}, \textit{customer innovation}, Business Intelligence, Corporate Performance Management e per le soluzioni \gls{IoT};
  \item \textbf{4WORDS:} specializzata in soluzioni \gls{B2B}, app e \gls{CRM}, ha l’obiettivo di far crescere le aziende grazie a soluzioni digitali dedicate: portale B2B, \gls{app} custom, app per la rete vendita e l’assistenza tecnica, ma anche realtà aumentata e \gls{PIM};
  \item \textbf{TCE:} si occupa di ottimizzare la fase di preventivazione e di acquisizione dell’ordine; \\
  Sviluppa il prodotto \gls{CPQg}, strumento essenziale ai fini della configurazione dell’offerta, 
  della gestione della trattativa e del recepimento del contratto, completo di tutti i contenuti documentali necessari.
  Attraverso uno strumento CPQ la forza vendite può configurare l’offerta più idonea in autonomia, in funzione delle specifiche esigenze del momento, senza preoccuparsi delle complesse logiche commerciali che la piattaforma gestisce in automatico;
  \item \textbf{DISCOVERY QUALITY:} produce una soluzione di Governance aziendale per gestire in modo efficace tutti i processi. 
  Un potente motore di workflow guida in modo preciso l’operatività del management e degli utenti, e inoltre misura le performance dell’impresa. \\
  Discovery Quality gestisce anche le principali normative internazionali e le metriche legate alla sostenibilità aziendale \gls{SDGs} e \gls{BCorp}.


\end{itemize}


\begin{figure}[!h] 
  \centering 
  \includegraphics[width=1\columnwidth]{organizzazione-azienda} 
  \caption{Le BU di {\azienda} ed i loro prodotti}
  \label{fig:organizzazione-azienda}
\end{figure}

\newpage

\section{Il team di sviluppo}
Il team di sviluppo in cui ho lavorato fa parte della BU \textit{JPA} (\textit{Process Management}) e non si
occupa di sviluppare un prodotto specifico, ma ha come obiettivo quello di fornire supporto a tutti i team di sviluppo
dell'azienda. \\
L'azione di supporto si concretizza nella creazione di strumenti che permettano di automatizzare e semplificare
i processi di sviluppo, come ad esempio la compilazione, il rilascio di un prodotto o lo sviluppo di uno nuovo.
Il team si occupa anche dello sviluppo di un \textit{\gls{frameworkg}} interno che permette di creare applicazioni web
in modo semplice e veloce e di fornire supporto per la gestione di repository e di strumenti di \textit{\gls{continuous_integrationg}}

\section{Strumenti utilizzati}
I principali strumenti per lo sviluppo da me utilizzati sono stati i seguenti:\\
\begin{itemize}
  \item \textbf{Intellij IDEA:} un ambiente di sviluppo integrato (\gls{IDEg}) per il linguaggio di programmazione Java. Fornisce strumenti e funzionalità avanzate per supportare lo sviluppo efficiente del codice, il debug e la testing. Con la sua interfaccia user-friendly e le potenti funzionalità, come l'analisi statica del codice e il refactoring intelligente, IntelliJ IDEA è scelto da molti sviluppatori per creare applicazioni Java professionali;
  \item \textbf{WebStorm:} un IDE per lo sviluppo di applicazioni web, che fornisce un'esperienza di sviluppo ottimale. Grazie alla sua integrazione con strumenti di supporto per lo sviluppo web, come \textit{Node.js}, \textit{Angular}, \textit{React}, WebStorm permette di sviluppare applicazioni web moderne con facilità;
  \item \textbf{Neo4j Desktop:} un programma che permette di installare e gestire database \textit{Neo4j} in modo semplice e veloce. Permette di creare e gestire più database, di monitorare le performance e di eseguire query;
  \item \textbf{Git:} un sistema di controllo versione distribuito, utilizzato per il versionamento del codice sorgente; 
  \item \textbf{Gradle:} un sistema di automazione open source che gestisce le dipendenze e permette di automatizzare il processo di compilazione, testing, pubblicazione e deployment di un software;
  \item \textbf{Docker:} un progetto open source che automatizza il deployment di applicazioni all'interno di contenitori software, fornendo un'astrazione aggiuntiva grazie alla virtualizzazione a livello di sistema operativo di Linux;
  \item \textbf{Bitbucket:} un servizio di hosting per progetti che utilizzano Git come sistema di controllo versione. Fornisce strumenti per la collaborazione e la gestione del codice sorgente;
  \item \textbf{Jenkins:}  un software open source che permette di automatizzare il processo di \textit{build}, testing e deployment di un software;
  \item \textbf{Angular:} un {\gls{frameworkg}} open source per lo sviluppo di applicazioni web, scritto in TypeScript. 
  Fornisce un'architettura \gls{mvvm} e permette di creare applicazioni web dinamiche e scalabili;
  \item \textbf{Jira:} un software di tracciamento dei bug e gestione dei progetti, che permette di pianificare, monitorare e rilasciare software di qualità;
  \item \textbf{Confluence:} un software di collaborazione che permette di creare, organizzare e discutere documenti di progetto;
\end{itemize}

I linguaggi utilizzati sono i seguenti:\\

\begin{itemize}
  \item \textbf{Java:} un linguaggio di programmazione ad alto livello, orientato agli oggetti e a tipizzazione statica, che permette di creare applicazioni web, desktop e mobile;
  \item \textbf{Javascript:} un linguaggio di programmazione ad alto livello, orientato agli oggetti e a tipizzazione dinamica, che permette di creare applicazioni web dinamiche;
  \item \textbf{TypeScript:} un super-set di Javascript che permette di aggiungere tipizzazione statica al linguaggio;
  \item \textbf{Groovy:} un linguaggio di programmazione che permette di scrivere codice che viene eseguito sulla \gls{JVM};
  \item \textbf{Chyper:} un linguaggio di query dichiarativo per grafi, utilizzato per interrogare database \gls{Neo4j};
\end{itemize}
\newpage
\section{Rapporto con l'innovazione}

{\azienda} ha come obiettivo l’innovazione delle aziende clienti per contribuire al loro progresso, 
agevolando la trasformazione digitale ed è specializzata nella progettazione e nella realizzazione di soluzioni integrate, 
a supporto della riorganizzazione di tutti i processi aziendali e professionali.\\
Per raggiungere questo obiettivo, l'azienda indirizza ogni anno dal 15 al 20\% del proprio fatturato all’attività di Ricerca e Sviluppo.\\
Uno dei punti di forza di {\azienda} è la capacità di cogliere le idee e i suggerimenti dei clienti, dei dipendenti, dei collaboratori e trarne ispirazione per sviluppare nuovi prodotti e nuove soluzioni