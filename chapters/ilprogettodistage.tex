\chapter{Il progetto di stage}
\label{cap:ilprogettodistage}

\section{Lo stage per \azienda}
Per \azienda lo stage è un'opportunità per entrare in contatto con studenti universitari e per valutare 
le loro capacità e competenze. Da anni l'azienda collabora con l'Università di Padova, ospitando
studenti per lo svolgimento di tesi di laurea e di stage.\\
Lo stage è un'esperienza formativa che permette allo studente di applicare le conoscenze teoriche acquisite 
durante il corso di studi.\\
Lo stage per \azienda è quasi sempre votato all'assunzione, infatti negli ultimi sono stati assunti più di 140 stagisti.\\
Uno dei vantaggi che offre lo stage all'azienda è la possibilità di lavorare a progetti innovativi, che
permettono di sperimentare nuove tecnologie che non avrebbero spazio in un progetto in corso e che potrebbero
essere utilizzate in futuro.\\

\section{La proposta di stage}
Uno dei progetti futuri di \azienda è quello di riuscire a modularizzare i propri prodotti, in modo da poterli
utilizzare in modo indipendente e da poterli integrare facilmente con altri prodotti.\\
Questo comporta una grande crescita di progetti e di dipendenze tra di essi, che devono essere tenute monitorate.\\
Per fare ciò ,\azienda all’interno del proprio dipartimento di ricerca e sviluppo, ha deciso si sviluppare uno strumento 
che permette di raccogliere, monitorare ed interrogare 
le dipendenze e le relative versioni agganciate alle release dei vari prodotti, ai fini di rispondere facilmente ai requisiti
 non funzionali in termini di sicurezza delle proprie soluzioni.\\
La proposta di stage è stata quella di sviluppare un \textit{\gls{prototipo}} di questo strumento, che permettesse di
raccogliere le informazioni riguardanti le dipendenze dal \textit{build system} utilizzato per la compilazione dei prodotti 
(\textit{\gls{Gradle}} e \textit{\gls{Npm}}) nel caso specifico.