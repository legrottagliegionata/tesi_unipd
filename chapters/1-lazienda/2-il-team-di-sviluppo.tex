\section{Il \textit{team} di sviluppo}

Durante il mio periodo in JPA (\textit{Process Management}), ho avuto l'opportunità di osservare da vicino il lavoro di un \textit{team} di sviluppo particolarmente versatile. 
Il \textit{team}, a differenza di quelli più tradizionali focalizzati su un singolo prodotto, aveva l'obiettivo di fornire supporto globale a tutti i \textit{team} di sviluppo dell'azienda.
\noindent Il supporto tecnico e analitico era la principale attività del \textit{team}, affrontando e risolvendo problemi complessi per agevolare il lavoro degli altri \textit{team}. 
Questo ruolo cruciale implicava l'identificazione e la soluzione di sfide tecniche, assicurando un ambiente di sviluppo efficiente e senza ostacoli. \\
\noindent Un altro aspetto centrale del lavoro del \textit{team} era la formazione. Con l'evolversi continuo delle tecnologie e degli strumenti, era essenziale mantenere i \textit{team} aggiornati. Di conseguenza, il \textit{team} organizzava regolarmente sessioni di formazione per condividere conoscenze e competenze su nuove tecnologie e metodologie di sviluppo. \\
\noindent In parallelo, il \textit{team} era impegnato in attività di ricerca e sviluppo, in particolare nello sviluppo di un \textit{\gls{frameworkg}} interno. Il framework, un insieme di librerie e strumenti per lo sviluppo di applicazioni \textit{web}, era progettato per rendere la creazione di applicazioni \textit{web} più semplice e veloce, contribuendo significativamente all'efficienza dello sviluppo \textit{software} in azienda. \\
\noindent L'automazione dei processi di sviluppo era un'altra area chiave, includeva l'automazione della compilazione, il rilascio dei prodotti e lo sviluppo di nuove funzionalità, riducendo il tempo e lo sforzo necessari per le operazioni di \textit{routine}. \\
\noindent La gestione dei \textit{\gls{repository}} di codice sorgente e il supporto all'uso di strumenti di \textit{\gls{continuous_integrationg}} erano compiti fondamentali del \textit{team}; assicurava una gestione efficace del codice e un'integrazione continua, elementi vitali per mantenere la qualità e l'affidabilità del \textit{software}. \\
\noindent Infine, il \textit{team} era responsabile dello sviluppo di un installatore per i prodotti dell'azienda basati sul \textit{\gls{frameworkg}} interno. Lo strumento semplificava il processo di installazione, rendendo i prodotti più accessibili agli utenti finali. \\
\noindent Il \textit{team} era composto da tre persone: uno \gls{Scrum Master} e due sviluppatori. In questo ambiente dinamico, i ruoli erano fluidi: gli sviluppatori svolgevano anche compiti di analisi e \textit{test}, e lo \gls{Scrum Master} partecipava attivamente alle analisi tecniche e funzionali. La struttura flessibile e collaborativa era essenziale per il successo del \textit{team} e per il supporto efficace fornito agli altri \textit{team} di sviluppo. \\
