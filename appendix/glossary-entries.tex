\newacronym[description={\glslink{POg}{Product Owner}}]
    {PO}{PO}{Product Owner}

% Acronyms
\newacronym[description={\glslink{apig}{Application Program Interface}}]
    {api}{API}{Application Program Interface}
    
\newacronym[description={\glslink{IoTg}{Internet of Things}}]
    {IoT}{IoT}{Internet of Things}

\newacronym[description={\glslink{ERPg}{Enterprise Resource Planning}}]
    {ERP}{ERP}{Enterprise Resource Planning}

\newacronym[description={\glslink{umlg}{Unified Modeling Language}}]
    {uml}{UML}{Unified Modeling Language}

\newacronym[description={\glslink{IDEg}{Integrated Development Environment}}]
    {ide}{IDE}{Integrated Development Environment}

\newacronym[description={\glslink{MVVMg}{Model-View-ViewModel}}]
{mvvm}{MVVM}{Model-View-ViewModel}

\newacronym[description={\glslink{CPQg}{Configure Price Quote}}]
{CPQ}{CPQ}{Configure Price & Quote}

\newacronym[description={\glslink{ECMg}{Enterprise Content Management}}]
{ECM}{ECM}{Enterprise Content Management}

\newacronym[description={\glslink{B2Bg}{Business to Business}}]
{B2B}{B2B}{Business to Business}

\newacronym[description={\glslink{BPMg}{Business Process Management}}]
{BPM}{BPM}{Business Process Management}

\newacronym[description={\glslink{frameworkg}{Framework}}]
{JVM}{JVM}{Java Virtual Machine}

\newacronym[description={\glslink{app}{Appplicazioni}}]
{app}{App}{Applicazioni}


%CRM
\newacronym[description={\glslink{CRMg}{Customer Relationship Management}}]
{CRM}{CRM}{Customer Relationship Management}

%PIM
\newacronym[description={\glslink{PIMg}{Product Information Management}}]
{PIM}{PIM}{Product Information Management}

%SDGs
\newacronym[description={\glslink{SDGs}{Sustainable Development Goals}}]
{SDGs}{SDGs}{Sustainable Development Goals}

%BCorp
\newacronym[description={\glslink{BCorp}{Benefit Corporation}}]
{BCorp}{BCorp}{Benefit Corporation}

%IT
\newacronym[description={\glslink{IT}{\textit{Information Technology}}}]
{IT}{IT}{Information Technology}

% Glossary entries
\newglossaryentry{apig} {
    name=\glslink{api}{API},
    text=Application Program Interface,
    sort=api,
    description={in informatica con il termine \emph{Application Programming Interface API} (ing. interfaccia di programmazione di un'applicazione) si indica ogni insieme di procedure disponibili al programmatore, di solito raggruppate a formare un set di strumenti specifici per l'espletamento di un determinato compito all'interno di un certo programma. La finalità è ottenere un'astrazione, di solito tra l'hardware e il programmatore o tra software a basso e quello ad alto livello semplificando così il lavoro di programmazione}
}

\newglossaryentry{umlg} {
    name=\glslink{uml}{UML},
    text=UML,
    sort=uml,
    description={in ingegneria del software \emph{UML, Unified Modeling Language} (ing. linguaggio di modellazione unificato) è un linguaggio di modellazione e specifica basato sul paradigma object-oriented. L'\emph{UML} svolge un'importantissima funzione di ``lingua franca'' nella comunità della progettazione e programmazione a oggetti. Gran parte della letteratura di settore usa tale linguaggio per descrivere soluzioni analitiche e progettuali in modo sintetico e comprensibile a un vasto pubblico}
}

\newglossaryentry{IDEg} {
    name=\glslink{ide}{IDE},
    text=IDE,
    sort=ide,
    description={L'acronimo IDE sta per "Integrated Development Environment" che in italiano si traduce come "Ambiente di Sviluppo Integrato". Un IDE è un software che fornisce strumenti e servizi integrati per facilitare ai programmatori lo sviluppo di software. Include spesso un editor di codice, strumenti per il debugging, e funzionalità per la gestione di progetti, tra gli altri.}
}

\newglossaryentry{continuous_integrationg} {
    name=Continuous integration,
    text=continuous integration,
    sort=continuous integration,
    description={in ingegneria del software, l'integrazione continua (\textit{continuous integration}) è una pratica che si applica in contesti in cui lo sviluppo del software avviene attraverso un sistema di versionamento. Consiste nell'allineamento frequente dagli ambienti di lavoro degli sviluppatori verso l'ambiente condiviso, al fine di rilevare tempestivamente eventuali errori di integrazione}
}

\newglossaryentry{MVVMg} {
    name=\glslink{mvvm}{MVVM},
    text=MVVM,
    sort=MVVM,
    description={è un pattern architetturale utilizzato nello sviluppo software per separare la logica di business dall'interfaccia utente. Consiste in tre componenti principali:\\
    Model: rappresenta i dati e la logica di business dell'applicazione.\\
    View: è l'interfaccia utente che visualizza le informazioni al utente.\\
    ViewModel: funge da ponte tra il Model e la View, gestendo la logica dell'interfaccia utente.\\
    MVVM facilita una separazione pulita delle preoccupazioni, rendendo il codice più organizzato e più facile da mantenere e testare.}
}


\newglossaryentry{frameworkg} {
    name=Framework,
    text=framework,
    sort=framework,
    description={Un \textit{framework} è una struttura concettuale e tecnologica predefinita che fornisce un modello standard su cui gli sviluppatori possono costruire applicazioni. Include librerie di codice, strumenti e linee guida che facilitano lo sviluppo, consentendo agli sviluppatori di concentrarsi sulla logica specifica dell'applicazione piuttosto che su dettagli di basso livello. Un framework può anche promuovere le buone pratiche di programmazione e ridurre la probabilità di errori.}
}

\newglossaryentry{Neo4j}{
  name=Neo4j,
  text=Neo4j,
  sort=Neo4j,
  description={Neo4j è un database orientato ai grafi, open source, sviluppato in Java da Neo Technology. Il database è implementato in Java e Scala.}
}

\newglossaryentry{Gradle}{
  name=Gradle,
  text=gradle,
  sort=Gradle,
  description={Gradle è un sistema di automazione open source che gestisce le dipendenze e permette di automatizzare il processo di compilazione, testing, pubblicazione e deployment di un software.}
}

\newglossaryentry{Npm}{
  name=Npm,
  text=npm,
  sort=Npm,
  description={Npm è un gestore di pacchetti per il linguaggio di programmazione JavaScript. È il gestore di pacchetti predefinito per l'ambiente di runtime JavaScript Node.js.}
}

\newglossaryentry{IoTg}{
  name=\glslink{IoT}{IoT},
  text=IoT,
  sort=IoT,
  description={L'Internet of Things (IoT) è un sistema di dispositivi interconnessi digitalmente, macchine, oggetti, animali o persone che sono forniti di identificatori univoci e la capacità di trasferire dati su una rete senza richiedere interazioni uomo-uomo o uomo-computer.}
}

\newglossaryentry{prototipo}{
  name=Prototipo,
  text=prototipo,
  sort=prototipo,
  description={Un prototipo è un esemplare o un modello di un prodotto o di un sistema che viene realizzato prima del prodotto finale.}
}
% Enterprise Resource Planning
\newglossaryentry{ERPg}{
  name=\glslink{ERP}{ERP},
  text=ERP,
  sort=ERP,
  description={Un sistema di pianificazione delle risorse aziendali (ERP) è un sistema di gestione che consente a un'organizzazione di utilizzare un sistema di applicazioni integrate per gestire l'attività aziendale e automatizzare molte funzioni back office relative alla tecnologia, ai servizi e ai processi umani.}
}

%B2B
\newglossaryentry{B2Bg}{
  name=\glslink{B2B}{B2B},
  text=B2B,
  sort=B2B,
  description={Business-to-business (B2B) è un modello di business che si riferisce alle transazioni commerciali tra due aziende, come quelle tra un produttore e un grossista o tra un grossista e un dettagliante.}
}

%CRM
\newglossaryentry{CRMg}{
  name=\glslink{CRM}{CRM},
  text=CRM,
  sort=CRM,
  description={Customer relationship management (CRM) è un approccio per gestire l'interazione di un'azienda con i clienti attuali e potenziali. Utilizza l'analisi dei dati sui clienti per migliorare le relazioni con i clienti, concentrarsi sulla customer retention e guidare le vendite.}
}

%PIM
\newglossaryentry{PIMg}{
  name=\glslink{PIM}{PIM},
  text=PIM,
  sort=PIM,
  description={Un sistema di gestione delle informazioni sui prodotti (PIM) è un insieme di strumenti e processi che un'azienda utilizza per gestire le informazioni sui prodotti necessarie per vendere e distribuire i propri prodotti a un acquirente finale.}
}

\newglossaryentry{ECMg}{
  name=\glslink{ECM}{ECM},
  text=ECM,
  sort=ECM,
  description={Enterprise content management (ECM) è un insieme di strumenti e strategie che consentono a un'organizzazione di acquisire, organizzare, archiviare e distribuire informazioni critiche per l'organizzazione.}
}

%CPQ
\newglossaryentry{CPQg}{
  name=\glslink{CPQ}{CPQ},
  text=CPQ,
  sort=CPQ,
  description={Configure Price Quote (CPQ) è un software che consente alle aziende di automatizzare alcuni dei processi più complessi e propensi agli errori nella vendita di prodotti e servizi.}
}

\newglossaryentry{cybersecurity}{
  name=Cybersecurity,
  text=Cybersecurity,
  sort=Cybersecurity,
  description={La cybersecurity è la pratica di proteggere i sistemi, le reti e i programmi da attacchi digitali. Questi attacchi sono generalmente mirati a accedere, modificare o distruggere informazioni sensibili; estorcere denaro ai utenti; o interrompere normali operazioni aziendali.}
}

\newglossaryentry{data_protection}{
  name=Data Protection,
  text=Data Protection,
  sort=Data Protection,
  description={La protezione dei dati è il processo di protezione delle informazioni da perdite, compromissione, attacchi o qualsiasi altra minaccia che possa compromettere la loro integrità.}
}

\newglossaryentry{BPMg}{
  name=\glslink{BPM}{BPM},
  text=BPM,
  sort=BPM,
  description={Business process management (BPM) è un approccio alla gestione delle operazioni aziendali che si concentra su allineamento tutti i processi con i desideri e le esigenze dei clienti.}
}

\newglossaryentry{Supply Chain}{
  name=Supply Chain,
  text=Supply Chain,
  sort=Supply Chain,
  description={La supply chain è la rete globale di tutte le organizzazioni coinvolte nella creazione e nella distribuzione di un prodotto o servizio.}
}

\newglossaryentry{Project Management}{
  name=Project Management,
  text=Project Management,
  sort=Project Management,
  description={Il project management è l'insieme di attività di pianificazione, organizzazione, gestione e controllo di un progetto.}
}

\newglossaryentry{analisti}{
  name=Analisti,
  text=analisti,
  sort=analisti,
  description={L'analista è una figura professionale che si occupa di analizzare i requisiti del cliente e di redigere la documentazione. Molte volte gli analisti sono anche sviluppatori e tester.}
}
% PO
\newglossaryentry{POg}{
  name=\glslink{PO}{PO},
  text=PO,
  sort=PO,
  description={Il Product Owner (PO) è una figura professionale che si occupa di gestire il progetto e di interfacciarsi con il cliente.}
}

% Scrum Master
\newglossaryentry{Scrum Master}{
  name=\glslink{Scrum Master}{Scrum Master},
  text=Scrum Master,
  sort=Scrum Master,
  description={Lo Scrum Master è una figura professionale che si occupa di gestire il team di sviluppo e di facilitare il processo di sviluppo.}
}

%repository
\newglossaryentry{repository}{
  name=Repository,
  text=repository,
  sort=repository,
  description={Un repository è un archivio di dati digitali.}
}