% Acronyms
\newacronym[description={\glslink{apig}{Application Program Interface}}]
    {api}{API}{Application Program Interface}

\newacronym[description={\glslink{umlg}{Unified Modeling Language}}]
    {uml}{UML}{Unified Modeling Language}

\newacronym[description={\glslink{IDEg}{Integrated Development Environment}}]
    {ide}{IDE}{Integrated Development Environment}

\newacronym[description={\glslink{MVVMg}{Model-View-ViewModel}}]
{mvvm}{MVVM}{Model-View-ViewModel}

\newacronym[description={\glslink{frameworkg}{Framework}}]
{JVM}{JVM}{Java Virtual Machine}

% Glossary entries
\newglossaryentry{apig} {
    name=\glslink{api}{API},
    text=Application Program Interface,
    sort=api,
    description={in informatica con il termine \emph{Application Programming Interface API} (ing. interfaccia di programmazione di un'applicazione) si indica ogni insieme di procedure disponibili al programmatore, di solito raggruppate a formare un set di strumenti specifici per l'espletamento di un determinato compito all'interno di un certo programma. La finalità è ottenere un'astrazione, di solito tra l'hardware e il programmatore o tra software a basso e quello ad alto livello semplificando così il lavoro di programmazione}
}

\newglossaryentry{umlg} {
    name=\glslink{uml}{UML},
    text=UML,
    sort=uml,
    description={in ingegneria del software \emph{UML, Unified Modeling Language} (ing. linguaggio di modellazione unificato) è un linguaggio di modellazione e specifica basato sul paradigma object-oriented. L'\emph{UML} svolge un'importantissima funzione di ``lingua franca'' nella comunità della progettazione e programmazione a oggetti. Gran parte della letteratura di settore usa tale linguaggio per descrivere soluzioni analitiche e progettuali in modo sintetico e comprensibile a un vasto pubblico}
}

\newglossaryentry{IDEg} {
    name=\glslink{ide}{IDE},
    text=IDE,
    sort=ide,
    description={L'acronimo IDE sta per "Integrated Development Environment" che in italiano si traduce come "Ambiente di Sviluppo Integrato". Un IDE è un software che fornisce strumenti e servizi integrati per facilitare ai programmatori lo sviluppo di software. Include spesso un editor di codice, strumenti per il debugging, e funzionalità per la gestione di progetti, tra gli altri.}
}

\newglossaryentry{continuous_integrationg} {
    name=continuous integration,
    text=continuous integration,
    sort=continuous integration,
    description={in ingegneria del software, l'integrazione continua (\textit{continuous integration}) è una pratica che si applica in contesti in cui lo sviluppo del software avviene attraverso un sistema di versionamento. Consiste nell'allineamento frequente dagli ambienti di lavoro degli sviluppatori verso l'ambiente condiviso, al fine di rilevare tempestivamente eventuali errori di integrazione}
}

\newglossaryentry{MVVMg} {
    name=\glslink{mvvm}{MVVM},
    text=MVVM,
    sort=MVVM,
    description={è un pattern architetturale utilizzato nello sviluppo software per separare la logica di business dall'interfaccia utente. Consiste in tre componenti principali:\\
    Model: rappresenta i dati e la logica di business dell'applicazione.\\
    View: è l'interfaccia utente che visualizza le informazioni al utente.\\
    ViewModel: funge da ponte tra il Model e la View, gestendo la logica dell'interfaccia utente.\\
    MVVM facilita una separazione pulita delle preoccupazioni, rendendo il codice più organizzato e più facile da mantenere e testare.}
}


\newglossaryentry{frameworkg} {
    name=framework,
    text=framework,
    sort=framework,
    description={Un \textit{framework} è una struttura concettuale e tecnologica predefinita che fornisce un modello standard su cui gli sviluppatori possono costruire applicazioni. Include librerie di codice, strumenti e linee guida che facilitano lo sviluppo, consentendo agli sviluppatori di concentrarsi sulla logica specifica dell'applicazione piuttosto che su dettagli di basso livello. Un framework può anche promuovere le buone pratiche di programmazione e ridurre la probabilità di errori.}
}

\newglossaryentry{Neo4j}{
  name=Neo4j,
  text=Neo4j,
  sort=Neo4j,
  description={Neo4j è un database orientato ai grafi, open source, sviluppato in Java da Neo Technology. Il database è implementato in Java e Scala.}
}