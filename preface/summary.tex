\cleardoublepage
\phantomsection
\pdfbookmark{Sommario}{Sommario}
\begingroup
\let\clearpage\relax
\let\cleardoublepage\relax
\let\cleardoublepage\relax

\chapter*{Sommario}

Il presente documento descrive il lavoro che ho svolto durante il periodo di \stage, della durata di trecentoventi ore, 
dal laureando {\myName} presso l'azienda {\azienda} \\
Gli obiettivi da raggiungere erano diversi.\\
In primo luogo era richiesto lo sviluppo di un \textit{plugin} Gradle per l'analisi statica delle dipendenze software di un progetto Gradle o npm;
in secondo luogo era richiesto di sviluppare dei servizi REST per il salvataggio dei risultati del \textit{plugin} e per effettuare la ricerca
delle vulnerabilità software note e, infine,
una \textit{web application} per la visualizzazione dei risultati.\\
Questo documento è strutturato in quattro capitoli principali:
\begin{enumerate}
    \item Il primo capitolo offre una panoramica del contesto aziendale e illustra gli strumenti utilizzati durante lo \stage;
    \item Nel secondo capitolo viene presentata la proposta di \stage, con un focus sugli obiettivi da raggiungere;
    \item Il terzo capitolo descrive in dettaglio le attività svolte durante lo \stage;
    \item Il quarto capitolo contiene riflessioni personali sull'esperienza lavorativa e sulle competenze acquisite.
\end{enumerate}

\noindent Durante la scrittura ho utilizzato termini in lingua inglese, in quanto è la lingua più utilizzata nel settore informatico, 
per riferirmi a concetti tecnici, essi sono stati evidenziati in \textit{corsivo}.\\
Per mettere in evidenza termini di particolare importanza ho utilizzato il \textbf{grassetto}.\\
Ho creato un glossario per chiarire il significativo di alcuni termini tecnici di non immediata comprensione, essi sono evidenziati in azzurro.\\
Ho allegato ad ogni figura un numero progressivo, in modo da poterla citare nel testo, ed una didascalia per descrivere il contenuto
e per citarne la fonte, se non è di mia creazione.\\


%\vfill

%\selectlanguage{english}
%\pdfbookmark{Abstract}{Abstract}
%\chapter*{Abstract}

%\selectlanguage{italian}

\endgroup

\vfill
