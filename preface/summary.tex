\cleardoublepage
\phantomsection
\pdfbookmark{Sommario}{Sommario}
\begingroup
\let\clearpage\relax
\let\cleardoublepage\relax
\let\cleardoublepage\relax

\chapter*{Sommario}

Il presente documento descrive il lavoro svolto durante il periodo di \textit{stage}, della durata di trecentoventi ore, dal laureando {\myName} presso l'azienda {\azienda} \\
Gli obiettivi da raggiungere erano diversi.\\
In primo luogo era richiesto lo sviluppo di un plugin gradle per l'analisi statica delle dipendenze software di un progetto gradle o npm;
in secondo luogo era richiesto di sviluppare dei servizi REST per il salvataggio dei risultati del plugin e per effettuare la ricerca
delle vulnerabilità software note e, infine,
una \textit{web application} per la visualizzazione dei risultati.\\
Questo documento è strutturato in quattro capitoli principali:
\begin{enumerate}
    \item Il primo capitolo offre una panoramica del contesto aziendale e illustra gli strumenti utilizzati durante lo \textit{stage}.
    \item Nel secondo capitolo viene presentata la proposta di \textit{stage}, con un focus sugli obiettivi da raggiungere.
    \item Il terzo capitolo descrive in dettaglio le attività svolte durante lo \textit{stage}.
    \item Il quarto capitolo contiene riflessioni personali sull'esperienza lavorativa e sulle competenze acquisite.
\end{enumerate}

\noindent Durante la scrittura ho utilizzato termini in lingua inglese, in quanto è la lingua più utilizzata nel settore informatico, per riferirmi a concetti tecnici, essi sono stati evidenziati in \textit{corsivo}.
Ho creato un glossario per chiarire il significativo di alcuni termini tecnici di non immediata comprensione, essi sono evidenziati in azzurro.
Ogni figura è accompagnata da una didascalia che ne descrive il contenuto ed è presente la relativa citazione, se non è di mia creazione.


%\vfill

%\selectlanguage{english}
%\pdfbookmark{Abstract}{Abstract}
%\chapter*{Abstract}

%\selectlanguage{italian}

\endgroup

\vfill
